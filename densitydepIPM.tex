%% \documentclass[preprint,5p,times,twocolumn,authoryear]{elsarticle}

%% Use the option review to obtain double line spacing
\documentclass[preprint,review,12pt,authoryear]{elsarticle}

%% Use the options 1p,twocolumn; 3p; 3p,twocolumn; 5p; or 5p,twocolumn
%% for a journal layout:
%% \documentclass[final,1p,times]{elsarticle}
%% \documentclass[final,1p,times,twocolumn]{elsarticle}
%% \documentclass[final,3p,times]{elsarticle}
%% \documentclass[final,3p,times,twocolumn]{elsarticle}
%% \documentclass[final,5p,times]{elsarticle}
%% \documentclass[final,5p,times,twocolumn]{elsarticle}

%% For including figures, graphicx.sty has been loaded in
%% elsarticle.cls. If you prefer to use the old commands
%% please give \usepackage{epsfig}

%% The amssymb package provides various useful mathematical symbols
\usepackage{amssymb}
\usepackage{amsmath}
\usepackage{latexsym}     
\usepackage{hyperref}
\usepackage{subcaption}
\usepackage{lineno}
\usepackage[margin=1in]{geometry}   

\usepackage{pstricks}
\usepackage{tikz} %required for wire diagram
\usetikzlibrary{shapes.geometric, arrows}
\usetikzlibrary{arrows,arrows.meta}
\usetikzlibrary{positioning} 
\usepackage{pst-node}
% add the following two lines to your document to get bigger arrows
\usetikzlibrary{arrows.meta}
\tikzset{>={Latex[width=3mm,length=3mm]}}
%% The amsthm package provides extended theorem environments
%% \usepackage{amsthm}

%% The lineno packages adds line numbers. Start line numbering with
%% \begin{linenumbers}, end it with \end{linenumbers}. Or switch it on
%% for the whole article with \linenumbers.
%% \usepackage{lineno}

\def\ds{\displaystyle}

\journal{Ecological Modeling}

\begin{document}

\linenumbers
\begin{frontmatter}


\title{An integral projection model for gizzard shad (\emph{Dorosoma cepedianum}) utilizing density-dependent age-0 survival}

\author{ J. Peirce$^{\rm 1,4}$,  G.  Sandland$^{\rm 3,4}$, B. Bennie$^{\rm 1}$, R.A. Erickson$^{\rm 2}$}
\address{
${\rm 1}$ University of Wisconsin - La Crosse, Mathematics \& Statistics Department\\ 
${\rm 2}$ U.S.G.S. Upper Mississippi Environmental Science Center\\ 
${\rm 3}$ University of Wisconsin - La Crosse, Biology Department\\
${\rm 4}$ River Studies Center} 

\begin{abstract}
%% Text of abstract
Gizzard shad (\emph{Dorosoma cepedianum}) are a common freshwater fish found throughout the central and eastern portions of North America. 
Within these areas, gizzard shad play a number of critical roles in the freshwater community. 
Because of this, it is important that we understand how gizzard shad populations respond to environmental changes and what these changes may mean for aquatic communities in general and fish assemblages in particular. 
Here we introduce an integral projection model for gizzard shad based on empirical data and include density-dependent survival in age-0 fish. 
Integral projection models (IPM) are a generalization of stage-based, matrix population models that have been used to describe a wide range of organisms. 
IPMs are a natural choice for gizzard shad since many aspects of their life cycle have been studied. 
In this paper, we compared model outcomes to empirical patterns reported for this fish species at a key location along the Illinois River. 
Results of our work suggest that this model could serve as an important tool for predicting gizzard shad population responses to changing environmental conditions, including those mediated through species invasions.
\end{abstract}

%%Graphical abstract
\begin{graphicalabstract}
\begin{figure*}
    \begin{center}
%\scalebox{.75}[.75]{
\begin{tikzpicture}[->,>=stealth',shorten >=1pt,auto,node distance=3cm,
  thick,
  main node/.style={rectangle,draw},
  box/.style = {draw=gray, very thick,
                            minimum height=11mm, text width=11mm, 
                            align=center},]
                              
\node[main node, align=center] (census_pres) at (0,3) {Census $t$ \\
        \includegraphics[width=0.12\textwidth]{adult.jpeg} \\
        $n(z,t)$};
\node[align=center] (repro) at (3.25,3) {Reproduction\\
        \includegraphics[width=0.12\textwidth]{adult.jpeg}\\
        };
\node (repro_in) at (2.3, 3)  {};
\node (repro_out) at (4.25, 3)  {};
\node (repro_down) at (3, 2.5)  {};
%\node[align=center] (surv) at (4.5,3) {\\
%        \includegraphics[width=0.15\textwidth]{adult.jpeg}\\
%        };
\node (grow_box) at (6.3,3.85) {Survival \& Growth};
\node[main node, align=center] (census_next) at (10,3) {Census $t+1$ \\
        \includegraphics[width=0.15\textwidth]{adult.jpeg}\\
        $n(z',t+1)$};
\node (repro_box) at (13,3.85)  {Reproduction};
\node[align=center] (eggs) at (3,0) {Eggs \\ $\bullet$ $\bullet$ $\bullet$ $\bullet$ \\
  \, $\bullet$ $\bullet$ $\bullet$ $\bullet$ \\ $\bullet$ $\bullet$ $\bullet$ $\bullet$};
  \node[align=center] (eggs_viable) at (5,0) {Viable \\ $\bullet$ $\bullet$ $\bullet$ \\
  \, $\bullet$ $\bullet$ \\ $\bullet$ $\bullet$ $\bullet$};
 \node[align=center] (age1) at (8,0) { Age-1\\
        \includegraphics[width=0.06\textwidth]{adult.jpeg}
         \includegraphics[width=0.04\textwidth]{adult.jpeg}\\
         \includegraphics[width=0.08\textwidth]{adult.jpeg}
        };
\node[] (next) at (13,3) {};

\path[->]
	(census_pres) edge node {} (repro_in)
	(repro_out) edge node {$s(z)G(z',z)$} (census_next)
	(census_next) edge node {} (next)
	(repro_down) edge node {$p_b \rm{egg}(z)$} (eggs)
	(eggs) edge node {$\nu$} (eggs_viable)
	(eggs_viable) edge node {$s_0(d)$} (age1)
	(age1) edge node[right, pos=0.3] {$C_1(z')$} (census_next);
	
\end{tikzpicture}
%}
\end{center}
 \caption{\small{Life cycle diagram and census points for pre-reproduction census of gizzard shad.}}
\end{figure*}

\end{graphicalabstract}

%%Research highlights
\begin{highlights}
\item Research highlight 1
\item Research highlight 2
\end{highlights}

\begin{keyword}
  population dynamics \sep
  fisheries \sep
  Mississippi River basin \sep
  population ecology \sep
  invasive species impact 
%% keywords here, in the form: keyword \sep keyword

%% PACS codes here, in the form: \PACS code \sep code

%% MSC codes here, in the form: \MSC code \sep code
%% or \MSC[2008] code \sep code (2000 is the default)

\end{keyword}

\end{frontmatter}

%% \linenumbers

%% main text


\section{Introduction}
Gizzard shad (\emph{Dorosoma cepedianum}) is a laterally compressed, deep-bodied fish species that occupies numerous aquatic systems throughout central, southern and eastern regions of the United States \citep{pierce1981aspects,vanni2005linking}.  
In more eutrophic habitats, such as reservoirs, gizzard shad can reach high abunadances and can come to dominate fish assemblages. 
Because of this, gizzard shad have the potential to influence freshwater systems in a number of ways. 
First, young shad often serve as a critical food source for many fish species, including those of commercial and recreational importance (such as walleye and largemouth bass)\citep{jester1972life}. 
Thus, this species can serve as an important trophic link within aquatic food webs.
Second, because detritus can serve as a primary food source throughout much of gizzard shad development (i.e. from the age-0 stage onward), these fish can transport nutrients
from benthic regions into pelagic habitats \citep{mather1995regeneration, schaus2000effects, vanni2005linking}. 
This process can result in an increase in the nutrients available to organisms within the water column leading to increases in phytoplankton biomass, algal blooms, and, due to these conditions, shifts in freshwater community structure \citep{aday2003direct, schaus2000effects}. 
Finally, the fact that detritus can comprise a substantial portion of gizzard shad diet also makes this species an important connection between terrestrial inputs and aquatic processes \citep{schaus2000effects}.
Given its potentially important role in aquatic ecosystems, interest has intensified in understanding how gizzard shad populations respond to environmental changes (both natural and anthropogenic) and what these changes may mean for freshwater communities in general and fish assemblages in particular.

Interactions within and between species in interconnected environments can have important consequences for fish populations across space and time \citep{thorp2006riverine}.  
For gizzard shad, previous work has suggested that fish densities can play an important role in both the growth and survival patterns observed in these fish populations.  
For example, \citep{buynak1992differential} reported an inverse relationship between densities and the lengths of age-0 gizzard shad.  Similarly \citep{welker1994growth} found that high densities of age-0 shad were negatively associated with both fish length and survival under both field and semi-natural conditions. Finally, \citep{michaletz2010overwinter} revealed that the densities of age-0 gizzard shad were negatively correlated with survival in two Missouri lakes.  
These patterns were attributed to intraspecific competition among young shad for prey (zooplankton) resources. Although intraspecific competition may be influencing life-history traits in subsequent stages of gizzard shad development, little work has actually been conducted to address this \citep{dicenzo1996relations}. 
There is also evidence that the densities of other co-occurring fish species (such as invasive carps) may also negatively influence aspects of gizzard shad biology, such as body condition \citep{irons2007reduced,love2018does}.

Although substantial empirical work on gizzard shad biology has accumulated over the decades, few if any studies have attempted to use these data to model the population dynamics of this species.
Work by \citet{catalano2010size, catalano2011whole} used empirically-based simulations of gizzard shad and focused on population-level responses.
The authors did investigate population-structure using fish lengths but did
not explore the effects of shad densities on population patterns.
Here, we introduce an integral projection model for gizzard shad based on empirical data with density-dependent survival in age-0 fish.
We then compare model outcomes to the dynamics reported for this fish species in the La Grange Station along the Illinois River.
The model itself could be an important tool for predicting gizzard shad population responses to changing environmental conditions, including those mediated through species invasions (i.e., silver and bighead carp).

\begin{table*}
  \caption{A summary of parameters, their biological meaning, and source for mean values.
  }
 \begin{center}
	\resizebox{\textwidth}{!}{
\begin{tabular}{p{6.5cm}|p{6cm}|c| p{6cm}  }\hline
     Parameter & Meaning (units) & Mean& Source \\\hline
Logistic survival probability function, $s(z)$ & & \\
  $\ds s_{\rm {min}}$ & minimum survival & 0.002 & \citep{bodola1955life} \\
  $\ds s_{\rm {max}}$ & maximum survival & $\ds 1-8.871K^{0.73}L_{\infty}^{-0.33}$ 
  & THEN 2015 \\ 
 $\ds \alpha_s$  & inflection point & 104.34& Estimated from LTRM dataset \\
  $\ds \beta_s$ & slope & -637.93 & Estimated from LTRM dataset \\\hline
  Growth function, $\ds G(z,z')$ & & \\
  $L_\infty$ & maximum length (in mm) & 394.30 & \citep{catalano2010size} \\
  $K_g$ & growth rate & 0.26 &  \citep{michaletz2017variation} \\
  $\sigma_g$ & growth standard deviation & 25 &  \citep{michaletz2017variation} \\\hline
  Normal distribution of length of age-1, $\ds C_1(z')$ & & \\
  $\mu_r$ & mean length of recruitment (in mm) & 105 & \citep{michaletz2017variation} \\
  $\sigma_r$  & standard deviation of length & 25 & \citep{michaletz2017variation} \\\hline
Eggs produced, egg(z) & & \\
$\ds \mbox{egg}_{\mbox{max}}$ & maximum number of eggs produced & 742,094 & Estimated from \citep{jons1997ovarian} \\
$\ds \alpha_e$ & inflection point & 314.44 & Estimated from \citep{jons1997ovarian} \\
$\ds \beta_e$ & slope & -7.12 & Estimated from \citep{jons1997ovarian} \\\hline
Survival of age-0, $s_0(d(t))$ & & \\
$\ds a_0$ & intercept & 0.27 & Estimated from \citep{michaletz2010overwinter} \\
$\ds b_0$ & decay rate & 0.003 & Estimated from \citep{michaletz2010overwinter} \\\hline
Spawning & & \\ 
 $\nu$  & probability that egg becomes viable & 0.002 & \citep{bodola1955life} \\
 $p_b$ & probability that female spawns & 0.90 &    \\ 
 \end{tabular} } \label{table:parameters}
     \end{center}
     \end{table*}    

\section{Model development}
\subsection{Gizzard shad life history}
Mature gizzard shad tend to mate between May and June, although this can vary based on water temperatures.. 
Males and females aggregate and then broadcast gametes into the surrounding water; fertilized eggs then settle and adhere to the bottom substrates. 
After a period of days, eggs hatch and fish develop from the larval stage to juveniles and eventually to adults. 
In many habitats, individuals can reach sexual maturity within a year. 
As gizzard shad mature, their diet preferences typically shift from phytoplankton and zooplankton early in development to detritus and zooplankton as adults. 
Given the large number of eggs produced by shad females ($>$ 300,000/year), there is evidence that intraspecific competition can be intense during early developmental stages in this species. 
The strength of this competition may then subside as fish transition to feeding on different food items during latter stages of development.   

\subsection{Equations}
We used an integral projection model to describe the life history of gizzard shad in the Upper Mississippi River (UMR) system. 
Integral projection models (IPM) were introduced by Easterling \citep{easterling2000size}, as a generalization of stage-based, matrix population models. 
IPMs have been used to describe a wide range of organisms \citep{ellner2016data, merow2014advancing, rees2014building} and have only recently be used to model fish populations \citep{erickson2017integral, liao2019dynamic, white2016fitting, pollesch2022developing}.
The availability of empirical size observation to aid with model parameterization makes IPM a natural choice for gizzard shad since many aspects of their life cycle have been studied. 
Specifically, functions used in our model incorporate data from studies on egg production and adult size,  survival of the age-0 stage and density, and nearly thirty years of length measurements in the main channel of the Upper Mississippi River system.

We assumed that variations among individual gizzard shad can be summarized by its length $z$ (in mm) ranging from the minimum possible length $L$ to the maximum value $U$. 
The state of the population at time $t$ (in years) is described by the length distribution $n(z,t)$. 
Specifically, for each time $t$, $n(z,t)$ is a smooth function of $z$ such that the number of individuals of length $z$ in the interval $[a,b]$ at time $t$ is $\ds \int_a^b n(z,t) \, dz$. 

Between times $t$ and $t+1$, individual gizzard shad may grow, die, and produce offspring that vary in length depending on the individuals current length (Figure \ref{life_cycle}). 
At time $t+1$ the population will have a length distribution defined by $n(z, t+1)$. 
For our model, we partition the life cycle of gizzard shad into two stages: 1) survival and growth, and  2) reproduction. 
For an individual of length $z$ at time $t$, $P(z',z)\Delta z$ is the probability that the individual is alive at time $t+1$, and its size is in the interval $[z', z' + \Delta z]$ (as with $n(z,t)$ this is an approximation that is valid for small $\Delta z$, and the exact probability is given by an integral like the one above). 
Similarly, $F(z',z)\Delta z$ is the number of new offspring in the interval $[z', z' + \Delta z]$ present at time $t+1$, per length-$z$ individual at time $t$.

\begin{figure*}
\label{life_cycle}
    \begin{center}
%\scalebox{.75}[.75]{
\begin{tikzpicture}[->,>=stealth',shorten >=1pt,auto,node distance=3cm,
  thick,
  main node/.style={rectangle,draw},
  box/.style = {draw=gray, very thick,
                            minimum height=11mm, text width=11mm, 
                            align=center},]
                              
\node[main node, align=center] (census_pres) at (0,3) {Census $t$ \\
        \includegraphics[width=0.12\textwidth]{adult.jpeg} \\
        $n(z,t)$};
\node[align=center] (repro) at (3.25,3) {Reproduction\\
        \includegraphics[width=0.12\textwidth]{adult.jpeg}\\
        };
\node (repro_in) at (2.3, 3)  {};
\node (repro_out) at (4.25, 3)  {};
\node (repro_down) at (3, 2.5)  {};
%\node[align=center] (surv) at (4.5,3) {\\
%        \includegraphics[width=0.15\textwidth]{adult.jpeg}\\
%        };
\node (grow_box) at (6.3,3.85) {Survival \& Growth};
\node[main node, align=center] (census_next) at (10,3) {Census $t+1$ \\
        \includegraphics[width=0.15\textwidth]{adult.jpeg}\\
        $n(z',t+1)$};
\node (repro_box) at (13,3.85)  {Reproduction};
\node[align=center] (eggs) at (3,0) {Eggs \\ $\bullet$ $\bullet$ $\bullet$ $\bullet$ \\
  \, $\bullet$ $\bullet$ $\bullet$ $\bullet$ \\ $\bullet$ $\bullet$ $\bullet$ $\bullet$};
  \node[align=center] (eggs_viable) at (5,0) {Viable \\ $\bullet$ $\bullet$ $\bullet$ \\
  \, $\bullet$ $\bullet$ \\ $\bullet$ $\bullet$ $\bullet$};
 \node[align=center] (age1) at (8,0) { Age-1\\
        \includegraphics[width=0.06\textwidth]{adult.jpeg}
         \includegraphics[width=0.04\textwidth]{adult.jpeg}\\
         \includegraphics[width=0.08\textwidth]{adult.jpeg}
        };
\node[] (next) at (13,3) {};

\path[->]
	(census_pres) edge node {} (repro_in)
	(repro_out) edge node {$s(z)G(z',z)$} (census_next)
	(census_next) edge node {} (next)
	(repro_down) edge node {$p_b \rm{egg}(z)$} (eggs)
	(eggs) edge node {$\nu$} (eggs_viable)
	(eggs_viable) edge node {$s_0(d)$} (age1)
	(age1) edge node[right, pos=0.3] {$C_1(z')$} (census_next);
	
\end{tikzpicture}
%}
\end{center}
 \caption{\small{Life cycle diagram and census points for pre-reproduction census of gizzard shad.}}
\end{figure*}

\subsubsection{Growth and survival}
We define $P(z'z) = s(z,T)G(z',z)$ where $s(z)$ is the adult annual survival probability and $G(z',z)$ describes the annual length transitions. 
We assumed that the survival function is a logistic function,
\begin{equation}\label{eq:surv}
s(z,T) = s_{\rm min} + \frac{s_{\rm max}-s_{\rm min}}{1+e^{\beta_s(\ln(z)-\ln(\alpha_s)),}}.
\end{equation}
with four parameters: the minimum survival rate $s_{\rm min}$; a maximum survival rate, $s_{\rm max}$; and intercept parameter, $\alpha_{s}$; and a slope parameter, $\beta_{s}$ \citep{bolker2008ecological}.  

We assumed that the growth function is a two-variable normal distribution centered around a modified von Bertalanffy function of the length at time t. 
The von Bertalanffy equation, commonly used to describe the length of a fish over time, is given by $\ds z(t) = L_{\infty} \left(1-e^{-K(t-t_0)} \right)$ where $L_\infty$ is maximum asymptotic length, $K$ is the growth rate, and $t_0$ is the initial time. 
The expected length in the next year
\begin{align*}
 z' =z(t+1) & =  L_{\infty} \left(1-e^{-K(t+1-t_0)} \right) =  L_{\infty} - L_{\infty}e^{-K(t-t_0)} e^{-K} \\
 & =   L_\infty - \left( z(t)-L_\infty \right) e^{-K} =   L_{\infty} \left(1-e^{-K} \right) + z(t)e^{-K}. 
 \end{align*}
Consequently, we assumed that 
%\begin{equation}\label{eq:grow}
$\ds G(z',z) = \mathrm{Prob}(z' \, | \,  z, L_{\infty}, K_g) = \mathrm{Normal PDF}(\mu_g, \sigma_g)$
%\end{equation}
where $K_g$ is the individual growth rate, $\mu_g =  L_{\infty} \left(1-e^{-K_g} \right) + z(t)e^{-K_g}$, and $\sigma_g$ is the standard deviation.

\subsection{Fecundity}
We define the fecundity kernel, 
\begin{equation}\label{eq:fecundity}
F(z', z) = p_b \, \mbox{egg}(z) \, \nu \, s_0(n(z,t)) \, C_1(z')
\end{equation}
where $p_b$ is the probability of reproducing, $\mbox{egg}(z)$ is the mean number of eggs produced, $\nu$ is the probability that an egg is viable, $s_0(n(z,t))$ is the density-dependent probability of surviving to age-1, and $C_1 (z')$ is the length distribution of new recruits at age-1 (when they are first censused).

We assumed that the mean number of eggs produced by females of a certain length is a three-parameter logistic function,
\begin{equation}\label{eq:egg}
\mbox{egg}(z) = \frac{\mbox{egg}_{\rm max}}{1+e^{\beta_e(\ln(z)-\ln(\alpha_e)),}}.
\end{equation}

The probability of survival of gizzard shad during their first year may depend on many factors \citep{michaletz2010overwinter} such as mean temperature, mean total length, the present density of age-0 fish.  
In this study, we focus only on the density factor and define the probability of survival of age-0 fish as the exponential function
\begin{equation}\label{eq:s0}
s_0(d(t)) = a_0 \, e^{-b_0 d(t)}
\end{equation}
where $a_0$ is the intercept, $b_0$ the decay rate, and $d(t)$ is the density at time $t$ of age-0 gizzard shad per 1000 m$^3$, 
\[ d(t) = 10^{-3} \int_L^U p_b \, \mbox{egg}(z) \, \nu n(z,t) \, dz. \]  

Finally, after computing the total number of viable eggs produced and survive to age-1 fish, we multiplied this number with a normal distribution of length,
$ \ds C_1 (z') =  \mathrm{Normal PDF} (\mu_r, \sigma_r)$ where $\mu_r$ is the mean length of age-1 gizzard shad and $\sigma_r$ is the standard deviation. 

\subsection{Dynamical model} 
%In traditional matrix population models, the state vector is multiplied on the left by a matrix to project the current measurement to its future value.  In an integral projection model, the projection matrix is replaces by an integral kernel $K(z',z)$ that projects the population forward in time.  
The population at time $t+1$ is the sum of the contributions from each individual alive at time $t$,
\begin{equation}\label{eq:IPM}
n(z',t+1) = \int_L^U K(z',z)n(z,t) \,dz,
\end{equation}  
where $K(z',z) = s(z) G(z',z) + F(z',z)$ and $[L,U]$ is the range of all possible lengths.

\section{Methods}
\subsection{Research area used for model parameterization}
The Long Term Resource Monitoring (LTRM) is a U.S. Army Corp of Engineers program that assess, and detects changes in the fundamental health and condition of the Upper Mississippi River System (UMRS) ecosystem.  
Key ecological components of aquatic vegetation, bathymetry, fish, land use/land cover, and water quality are monitored in the natural floodplain between the head of navigation at Minneapolis, Minnesota and the confluence with the Ohio River at Cairo, Illinois. 
The LTRM fish collection methodology includes a multiple gear approach (netting and electrofishing) to monitor the general fish community in six study pools/reaches through time \citep{gutreuter1995long}.
Methodology, protocols and modifications to the LTRM can be found in \cite{gutreuter1995long}, and \cite{ickes2002evaluation}. 
With the exception of La Grange Reach, five (Pools 4, 8, 13, 26, and the Open River Reach) of the six study sites correspond to field stations located along the main channel of the river (Figure \ref{fig:field_stations}).
Total length of gizzard shad in the main channel of the UMRS has been recorded since 1989 with approximately 3000 random collections each year (Figure \ref{fig:LTRMmain}). 

\begin{figure}
\centering
\begin{subfigure}[b]{.43\textwidth}  
  \includegraphics[width=.6\textwidth]{figures/field_stations.png}
  \caption{}
  \label{fig:field_stations}
\end{subfigure}
\begin{subfigure}[b]{.43\textwidth} 
   \includegraphics[width=1.1\textwidth]{figures/LTRMmain.png}
   \caption{}
   \label{fig:LTRMmain}
\end{subfigure}
\caption{(a) Site source of image.  (b) Length frequency of sampled gizzard shad in the main channel of the Upper Mississippi River system.}
\end{figure}    

\subsection{Research area used for model validation}
The 129 km long La Grange Reach is located between La Grange Lock and Dam (L\&D) and Peoria L\&D on the Illinois River, U.S., and is approximately midway between the Mississippi River and Lake Michigan. 
The Illinois River is a major tributary of the Mississippi River, draining nearly two-thirds of the state of Illinois. 
Along with the main channel of the UMRS, the fish community of La Grange Reach has been monitored by LTRM from 1990 to the present, with approximately 500 random collections each year from 15 June to 31 October. 

The location of La Grange Reach has two important features motivation its choice for the methods of our study. 
First, we were able to parameterized the IPM using data collected from the main channel of the UMRS.  
As a part of Illinois River and UMRS, La Grange Reach is upstream and relatively independence from main channel of the Mississippi River.  
In recent years, there have been concerns with the threatening introduction of invasive carp to the Great Lakes.  
As a consequence, the impact of invasive carp on the native fish populations in the pools leading to the Great Lakes have received an elevated level of attention.  
Understanding the population dynamics of gizzard shad in La Grange Reach, may make it easier, in the future, to assess the impacts carp has on native fish populations.

\subsection{Parameterization}
\subsubsection{Fecundity and recruitment}
The maturity of female gizzard shad begin at approximately 140 mm and the number of eggs that are produced increases as the females increase in size \citep{jons1997ovarian}. 
The logistic parameters for the mean number of eggs produced by females of a certain length were obtained by fitting the three-parameter logistic function (Equation \ref{eq:egg}) to the data for batch fecundity versus length \citep{jons1997ovarian} (Figure \ref{fig:eggs}). 
The survival to age-0 was assumed to be dependent on the density of age-0 gizzard shad (Figure \ref{fig:surv_age0}).  
Parameters for the exponentially decaying age-0 survival function (Equation \ref{eq:s0}) were determined by fitting equation \ref{eq:s0} to the survival means for 2003-2007 cohorts of gizzard shard in five Missouri reservoirs \citep{michaletz2010overwinter}.
To complete the recruitment process we assign a length to the recruited individuals by simulating a Gaussian random variable with mean $\mu_c$ and standard deviation $\sigma_c$.
These parameters for the size distribution of age-1 fish were gleaned from a study of gizzard shad located in large impoundments \citep{michaletz2017variation} and the historic 1990-2020 LTRM dataset of the main channel of the UMRS.

\begin{figure}
\centering
\begin{subfigure}[b]{.43\textwidth}
  \includegraphics[width=\textwidth]{figures/eggs.png}
  \caption{}
  \label{fig:eggs}
\end{subfigure}
\begin{subfigure}[b]{.43\textwidth}
  \includegraphics[width=\textwidth]{figures/age0surv.png}
  \caption{}
  \label{fig:surv_age0}
\end{subfigure}
\caption{(a) Mean number of eggs produced by female gizzard shad $\mbox{egg}(z)$. Data from \citep{jons1997ovarian}.  (b) Age-0 density-dependent survival function $s_0(d)$. Data from \citep{michaletz2010overwinter}. }
\label{fig:fecundity}
\end{figure}    

\subsubsection{Growth and survival of adults}
The parameters for the growth function were chosen as the mean values published on a study of gizzard shad located in large impoundments \citep{michaletz2017variation}. 
The survival rate of adults gizzard shad dependent on their length is not well documented and required us to make some additional modeling assumptions.  
We assume that the probability of adult survival is related to the length by a four-parameter logistic function (Equation \ref{eq:surv}).
An investigation of gizzard shad in Lake Eerie (\cite{bodola1955life}) provided the minimum and maximum survival rate of adults. 
Based on the observed length distributions of gizzard shad sampled from the main channel of the Mississippi River (Figure \ref{fig:LTRMmain}), we assumed that solutions of our model will exhibit periodic behavior every 8-9 years.   
We used a least squares method to estimate the $\alpha_s$ and $\beta_s$ parameters that 
minimized the total square-distance between the (observed) pre-carp LTRM length distribution in the main channel of the UMRS and (predicted) model equilibrium, $n(z,t)$ during a 8-year period 100 years after initialization.  
The slope parameter $\beta_s$ was found to be large in magnitude resulting in a primarily two-valued survival probability.  
Gizzard shad less than $\alpha_s$ mm in length have a very low survival rate ($\ds s_{\rm {min}}$) while lengths larger than  $\alpha_s$ mm have the maximum survival rate $\ds s_{\rm {max}}$. 
This survival pattern has been reported for a number of fish species and can arise for a number of reasons including feeding attributes in predators (i.e. gape limitations, reduced abilities to catch prey, etc.)(Pepin et al., 1992; Nowlin et al., 2006). 
\section{Analysis and results}
We numerically solved the integral model using the Midpoint Rule with large approximating matrices \citep{burden2005numerical}. 
The Midpoint Rule has been commonly used for integral projection models because of its simplicity and effectiveness \citep{ellner2006integral, ramula2009integral,  merow2014advancing}. 
During the course of model development, we explored different step sizes for the Midpoint Rule and found that about 50 points provided numerically stable results. 
We integrated over lengths from 0 mm to 500 mm. 
The upper limit was chosen based upon numerical stability and consistency of the system (e.g., avoiding eviction or the loss of individuals due to numerical errors \citep{williams2012avoiding}). 

\subsection{Initial conditions}  We assumed that the initial density of gizzard shad was $d_0 = 964.7$, the annual average density of gizzard shad observed in La Grange Reach from 1993-2019.  
The probability of an individual being length $z$ at time $t=0$  was assumed to be normally distributed with mean $0.5L_\infty$ and standard deviation $\sigma_0 = 30$.  
As a result, we initialized our model with length distribution
\begin{equation}\label{eq:n}
 n(z,0) = d_0 \mbox{Norm} (0.5 L_\infty, \sigma_0) = 964.7 \mbox{Norm} (166, 30). 
 \end{equation}

The model was coded in R \citep{R} and the scripts are published on JP github page \verb+https://github.com/jppeirce+.

\subsubsection{Survival of age-0 cohort} \label{sec:survival}
The dependence on survival of next generation of age-0 fish on the present density of age-0 fish strongly influences the density, at all ages, of gizzard shad within the population. 
When the fish density is large, there may be more fish at longer lengths and consequently a greater number of eggs produced.  
More eggs leads to a higher density of age-0 fish and reflectively a reduction in the survival to age-1.  
If this reduced survival continues for a few years, the overall density of fish may decline and there may not be as many larger fish reproducing.  
If fewer eggs are spawned, there are less age-0 fish and the reduced density leads to a better survival probability.  
For these years we would expect the overall density to increase.  
This cycle of oscillation continues, reflected in our model by the time-dependent survival probability of age-0 recruits(Figure \ref{fig:age0time}).

\begin{figure}
\centering
%\begin{subfigure}[b]{.43\textwidth}
  \includegraphics[width=.4\textwidth]{figures/Figure2a.pdf}
   \caption{}
  \label{fig:age0time}
%\end{subfigure}
%\begin{subfigure}[b]{.43\textwidth}
%   \includegraphics[width=\textwidth]{figures/Figure2b.pdf}
%     \caption{}
%\label{fig:lambda}
%\end{subfigure}
\caption{Survival probability of age-0 gizzard shad.}
%\caption{(a) Graph of $p_b(z)$ (b) Graph of $\mbox{egg}(z)$, (c) Density-dependent survival of age-0 gizzard shard.}
%\label{fig:fecundity}
\end{figure}    

\subsubsection{Periodic orbit and validation with external dataset}
The total number of gizzard shad in our simulation reached a stable periodic orbit (Figure \ref{fig:ntotal}) within 50 years. 
The length distributions during the periodic orbit (Figure \ref{fig:period}) has similarities to length observations from La Grange found in the LTRM fish dataset (Figure \ref{fig:LTRMlg}). 
\begin{figure}
\centering
\begin{subfigure}[b]{.43\textwidth}
  \includegraphics[width=\textwidth]{figures/ntotal.png}
   \caption{}
  \label{fig:ntotal}
\end{subfigure}
\begin{subfigure}[b]{.43\textwidth}
   \includegraphics[width=\textwidth]{figures/period.png}
     \caption{}
\label{fig:period}
\end{subfigure}
\caption{(a) The total number of gizzard shad in La Grange Reach predicted by IPM for first 100 years. (b) Simulated length distributions during an 8 year interval of time (approximately 1 period of the the total density function) }
\end{figure}    
In addition. the average simulated length distribution compares favorably to the length frequencies of observations from La Grange. (Figure \ref{fig:lagrange}). 
\begin{figure}
\centering
\begin{subfigure}[b]{.43\textwidth}
  \includegraphics[width=\textwidth]{figures/LTRMlg.png}
   \caption{}
  \label{fig:LTRMlg}
\end{subfigure}
\begin{subfigure}[b]{.43\textwidth}
   \includegraphics[width=\textwidth]{figures/lagrange.pdf}
     \caption{}
\label{fig:lagrange}
\end{subfigure}
\caption{(a) Length frequencies of sampled gizzard shad in La Grange for each year, 1992-2020. (b) Length frequencies of sampled gizzard shad in La Grange from 1992-2020 compared with average (over 1 period) simulated length frequency.}
\end{figure}    

%We notice that the peak frequencies are near the same length with the model predicting slightly more adults lengths and fewer juvenile lengths than the observations. 

After simulating an additional 50 years, we fit a periodic function to the annual density of gizzard shad and determined a period of approximately 8.5 years. 
Figure \ref{fig:period} illustrates the periodic length dynamics within the gizzard shad population during a 8-year window of the periodic orbit.  
The ebb and flow of the frequency of the age-1 cohort are associated with the density-dependent survival function for age-0 fish as explained in Section \ref{sec:survival}.
In addition, Figure \ref{fig:period} captures the cohort length dynamics of the gizzard shad. 
Specifically, the peak length frequency of age-1 fish in Year 1 near $z = 105$ mm is seen as a peak in Year 3 near 185 mm, a peak near 305 mm in Year 3, before becoming nearly unidentifiable in the graphs of Year 7 and 9.  
This is consistent with the lifespan reported for gizzard that can range, on average, between 4 and 6 years (CITATION). 
The change in the location of these relative maximum influences the number of eggs being produced and consequently the density of age-0 fish the next year. 
An increase in the frequency of larger adults leads to a delayed-reduction in the age-1 cohort as a result of our  assumption of density-dependent survival function.

%\subsection{Time evolution of the initial state}
%[NOT NEEDED - NOW THAT WE HAVE PERIODIC GRAPH TO DISCUSS]
%Starting with $n(z,0)$ defined by Equation \ref{eq:n}, we computed the length distribution for 4 years.  
%In year 1, there are two relative maximum frequencies corresponding to the recruitment of age-1 fish and the survival of the adults.
%The greatest value of $n(z,1)$ during year one is centered at the mean length of recruitment (Figure \ref{fig:length_den}). 
% \begin{figure*}
%\centering
%\begin{subfigure}[b]{.43\textwidth}
%  \includegraphics[width=\textwidth]{figures/initial_sim_den.png}
%\caption{}
%\label{fig:length_den}
%\end{subfigure}
%\begin{subfigure}[b]{.43\textwidth}
%\includegraphics[width=\textwidth]{figures/initial_sim.png}
%\caption{}
%\label{fig:length_dist}
%\end{subfigure}
%\caption{Simulated (a) density and (b) frequency of lengths of gizzard shad. }
%\end{figure*}    
%In the following years, the fluctuation in the maximum value is likely the result the density-dependent survival function for age-0 fish as explained in Section \ref{sec:survival}. 
%
%Figure \ref{fig:length_den} also demonstrates the length dynamics within the gizzard shad population.  
%The peak length frequency of age-1 fish in $n(z,1)$ near $z = 105$ mm is seen as a peak in $n(z,2)$ near 180 mm before becoming unidentifiable in the graph of $n(z,3)$.  
%This is consistent with the average lifespan of 4-6 years (CITATION). 


\section{Discussion}

%\begin{figure*}
%\centering
%  \includegraphics[width=.8\textwidth]{figures/LTRMgraph.pdf}
%\caption{LTRM}
%\label{fig:LTRM}
%\end{figure*}    

Gizzard shad are common in freshwater systems of North America where they are important components fo the freshwater community. 
Given this, it is important to understand how populations of this species vary with changing environmental circumstances, such as the occurrence of invasive species. 
Herein we use both LTRM data and parameters gleaned from other empirical studies to develop an integral projection model for gizzard shad and test model outputs against patterns reported from a well-studied population of this species from the Illinois River.   

After fitting two adult survival parameters using LTRM data from the main channel of the Mississippi River, we compared a simulated length distribution with LTRM data collected from the La Grange Reach of the Illinois River from 1992 to 2020. 
The resulting simulated length distributions (Figure \ref{fig:period}) reflected a number of the patterns observed in gizzard shad captured from La Grange over the collection period (Figure \ref{fig:LTRMlg}). 
Specifically, the two peaks in Year 1 are similar to the observations made in 1997.  
In 1997, there is wider range of mid-length and longer length gizzard shad and relative smaller amount of age-1 fish recorded.  
In the next year, there was a higher frequency of age-1 fish in both the LTRM data and the simulations (Year 2).

We notice that the peak frequencies are near the same length with the model predicting slightly more adults lengths and fewer juvenile lengths than the observations. 
This may be explained by gear bias.
The methods used to capture fish make it more likely to record longer lengths ($>200$mm). 
Studies [CITATION] suggest that due to the environmental stochasticity and other effects, smaller recruitment fish densities can fluctuate annually and be difficult to measure accurately. 

While our model uses age-0 density to effect age-0 survival, we assumed constant viability which may be sensitive to the external factors mentioned above.
In addition, the location of the maximum length and the variation in the of length of new recruits recorded in the LTRM data, suggests that there may be smaller age-0 fish in La Grange Reach than in study location \citep{michaletz2017variation} used to parameterize the model.  

While not all observed length distributions are reflected in the simulations, the similarity in the average length distributions (Figure \ref{fig:lagrange}) suggest that our IPM could serve as a tool for predicting length distributions for gizzard shad across different study areas. (LAST SENTENCE NEEDS A LOT MORE CLARITY)

Gaining an understanding of how length distributions of gizzard shad emerge under density-dependent survival in the age-0 class will serve as a foundation for investigating density effects at subsequent stages in the life cycle.  
In addition, this single-species model could also be expanded to incorporate interspecific interactions between gizzard shad and species such as invasive carp, which appear to negatively impact gizzard shad life-histories through competition for food resources.   

<<<<<<< HEAD
\begin{enumerate}
\item Summary of findings and key discussion points
  \begin{enumerate}
  \item Comparison to empirical data [X]
  \item Deviations from empirical data
  \item discussion about sameness and differences 
  \item Sources of density within our model
  \end{enumerate}
\item Comparison to existing literature
  \begin{enumerate}
  \item Talk about Matt's work \citep{catalano2010size,
      catalano2011whole}
  \item Broader need for models such as this
  \end{enumerate}
\item Implications for management of species
  \begin{enumerate}
  \item Invasive species
  \item Impact of size on harvest
  \item impact of size on movement
  \end{enumerate}
\item Future ideas to explore
  \begin{enumerate}
  \item Multi-species model
  \item Spatial impacts
  \item Climate on density
  \item Changing climate scenarios
  \end{enumerate}
\end{enumerate}%
=======
%\begin{enumerate}
%\item Summary of findings and key discussion points
 % \begin{enumerate}
 % \item Comparison to empirical data [X]
  %\item Deviations from empirical data
  %\item discussion about sameness and differences 
  %\item Sources of density within our model
  %\end{enumerate}
%\item Comparison to existing literature
  %\begin{enumerate}
  %\item Talk about Matt's work \citep{catalano2010size,
   %   catalano2011whole}
 % \item Broader need for models such as this
  %\end{enumerate}
%\item Implications for management of species
  %\begin{enumerate}
  %\item Invasive species
  %\item Impact of size on harvest
  %\item impact of size on movement
  %\end{enumerate}
%\item Future ideas to explore
  %\begin{enumerate}
  %\item Multi-species model
  %\item Spatial impacts
  %\item Climate on density
  %\item Changing climate scenarios
  %\end{enumerate}
%\end{enumerate}
>>>>>>> 03ab37f8bfb23321b5413edc6f69595da2687925

\section{Acknowledgments}

These data are a product of the U.S. Army Corps of Engineer's Upper Mississippi 
River Restoration Program (UMRR) Long Term Resource Monitoring (LTRM) element 
implemented by the U.S. Geological Survey in collaboration with the five 
Upper Mississippi River System (UMRS) states of Illinois, Iowa, Minnesota, 
Missouri, and Wisconsin.
The U.S. Army Corps of Engineers (Corps) 
provides guidance and has overall program responsibility.

We thank the U.S.~Geological Survey  Biothreats program and Great
Lakes Restoration Initiative for funding.
In addition, research was supported by NSF-DMS Grant \#1852224, ``REU Site: Ecological Modeling of the Mississippi River Basin''. 


 \bibliographystyle{elsarticle-num-names} 
 \bibliography{gizshad}



\end{document}

\endinput
%%
%% End of file `elsarticle-template-num-names.tex'.
